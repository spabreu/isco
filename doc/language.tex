% $Id$

% -----------------------------------------------------------------------------
%  ISCO is Copyright (C) 1998-2001 Salvador Abreu
%  
%     This program is free software; you can redistribute it and/or
%     modify it under the terms of the GNU General Public License as
%     published by the Free Software Foundation; either version 2, or
%     (at your option) any later version.
%  
%     This program is distributed in the hope that it will be useful,
%     but WITHOUT ANY WARRANTY; without even the implied warranty of
%     MERCHANTABILITY or FITNESS FOR A PARTICULAR PURPOSE.  See the GNU
%     General Public License for more details.
%  
%     You should have received a copy of the GNU General Public License
%     along with this program; if not, write to the Free Software
%     Foundation, Inc., 59 Temple Place - Suite 330, Boston, MA
%     02111-1307, USA.
%  
%  On Debian GNU/Linux systems, the complete text of the GNU General
%  Public License can be found in `/usr/share/common-licenses/GPL'.
% -----------------------------------------------------------------------------

\section{The ISCO language}
\label{sec:isco-language}

An ISCO program is a regular Prolog program augmented with a syntax for
describing:
\begin{itemize}
\item ISCO predicates.
\item ISCO goals.
\end{itemize}

\subsection{ISCO predicate syntax}
\label{sec:pred-syntax}

An ISCO predicate is understood to be similar to a Prolog predicate.  However,
its actual implementation may differ because some predicates may rely upon an
external storage mechanism.  Syntactically, ISCO predicates are different from
regular predicates because:
\begin{itemize}
\item Arguments are typed.
\item There may be constraints that specific arguments are automatically
  required to satisfy.
\item There's a built-in notion of update for ISCO predicates.
\end{itemize}

The syntax for an ISCO program will be given incrementally.  We start with
figure~\ref{fig:syntax:program} in which a program is shown to be a sequence
of ISCO definitions, interspersed with Prolog code (either clauses or
directives).  The notation used to express the ISCO syntax is the standard one
for grammars, with the possibility to define ...


\subsubsection{Programs and Definitions}
\label{sec:syntax:programs-definitions}

\begin{figure}[htb]
  \centering
  \begin{eqnarray*}
    \mathsf{PROGRAM} & \rightarrow & \mathsf{DEF} \; \mathsf{PROGRAM} \\
    & | & \mathsf{CLAUSE} \; \mathsf{PROGRAM} \\
    & | & \mathsf{DIRECTIVE} \; \mathsf{PROGRAM} \\
    & | & \varepsilon \\
    \\
    \mathsf{DEF} & \rightarrow & \mathsf{CLASS\_DEF} \;|\;
    \mathsf{SEQUENCE\_DEF} \;|\; \mathsf{EXTERNAL\_DEF}
  \end{eqnarray*}
  \caption{ISCO program syntax}
  \label{fig:syntax:program}
\end{figure}
In figure~\ref{fig:syntax:program} it can be seen that ISCO definitions
(non-terminal \textsf{DEF}) may take on three forms:
\begin{enumerate}
\item A \emph{class} definition (\textsf{CLASS\_DEF}), which is further
  described in figure~\ref{fig:syntax:class}.
\item A \emph{sequence} definition (\textsf{SEQUENCE\_DEF}), which can be used
  to implement global counters in an application, which is described in
  figure~\ref{fig:syntax:sequence}.
\item An \emph{external} declaration (\textsf{EXTERNAL\_DEF}) which provides
  ISCO with the necessary information to access an external data source or
  sink, such as an ODBC-driven database or an LDAP directory.  External
  definitions are described in figure~\ref{fig:syntax:external}.
\end{enumerate}

\subsubsection{Class definitions}
\label{sec:syntax:class-definitions}

The syntax for class definitions is given in figure~\ref{fig:syntax:class}
(see page~\pageref{fig:syntax:class}).
\begin{figure}[htb]
  \centering
  \begin{eqnarray*}
    \mathsf{CLASS\_DEF} & \rightarrow & \mathsf{HEAD} \; \mathsf{BODY} \;
    \mathsf{REST} \\
    \\
    \mathsf{HEAD} & \rightarrow & \mathsf{sequence}(\textsf{CLASS\_ATTR}) \;
    \keyword{class} \; \mathsf{CLASS\_NAME} \token{.} \\
    & | & \mathsf{sequence}(\textsf{CLASS\_ATTR}) \; \keyword{class} \;
    \mathsf{CLASS\_NAME} \token{:} \mathsf{SUPERCLASS\_NAME} \token{.} \\
  \end{eqnarray*}
  \caption{ISCO class definition syntax}
  \label{fig:syntax:class}
\end{figure}
The \textsf{HEAD} production indicates how a class is introduced.  The
sequence of \textsf{CLASS\_ATTR}s is optional and may be used to indicate
certain attributes for the class being defined.  These are comma-separated
terms and are enumerated in figure~\ref{fig:syntax:class-attr}, and may be
summarized as follows:
\begin{itemize}
\item \textsf{static}.  A class with the \textsf{static} attribute may not
  have new values appended to its set of solution tuples.  Typically such
  classes will be directly implemented as compiled Prolog code.
\item \textsf{mutable}.  A class with this attribute may be accessed in all
  three modes (see~)  \TBD
\item \textsf{external(ID, EXT\_ID)}.  This tells ISCO that the class being
  declared is in fact stored as \textsf{EXT\_ID} in external database
  \textsf{ID}.  \textsf{EXT\_ID} is a term to be interpreted in the external
  database's context.  This mechanism allows for the reuse of existing
  databases, avoiding name clashes without having to rename the relations.
\item \textsf{computed}.  This indicates that the class being declared is
  implemented as a set of \emph{rules}, ie.~as a regular Prolog predicate.
  The rules themselves are expected to follow the class definition. (see~\TBD)
\end{itemize}
\begin{figure}[htb]
  \centering
  \begin{eqnarray*}
    \mathsf{CLASS\_ATTR} & \rightarrow & \token{static} \;|\; \token{mutable}
    \;|\; \token{computed} \;|\; \token{external}\token{(} \mathsf{ID}
    \token{,} \mathsf{EXT\_ID} \token{)} \;|\; \varepsilon
  \end{eqnarray*}
  \caption{ISCO class attributes}
  \label{fig:syntax:class-attr}
\end{figure}

\subsubsection{Class body}
\label{sec:syntax:class-body}

The class body syntax is given by figure~\ref{fig:syntax:class-body} (see
page~\pageref{fig:syntax:class-body}) and is basically made up of a (possibly
empty) sequence of field definitions.
\begin{figure}[htb]
  \centering
  \begin{eqnarray*}
    \mathsf{BODY} & \rightarrow & \mathsf{FIELD\_DEF} \\
    & | & \mathsf{BODY} \; \mathsf{FIELD\_DEF} \\
    & | & \varepsilon \\
    \\
    \mathsf{FIELD\_DEF} & \rightarrow & \keyword{field} \;
    \mathsf{NAME} \token{:} \mathsf{TYPE} \token{.} \mathsf{FIELD\_ATTRS} \\
    & | &  \keyword{field} \; \mathsf{FIELD\_NAME} \token{.}
    \mathsf{FIELD\_ATTRS} \\
    & | &  \keyword{field} \; \mathsf{NAME} \token{:} \mathsf{CLASS\_NAME}
    \token{.} \mathsf{FIELD\_NAME} \token{.} \mathsf{FIELD\_ATTRS} \\
  \end{eqnarray*}
  \caption{ISCO class body definition syntax}
  \label{fig:syntax:class-body}
\end{figure}
The body of a class may be empty, in which case it will have no fields (if it
is a base class) or only the fields already in its superclass (if it inherits
from another class).  The \textsf{FIELD\_NAME} symbol is a Prolog
``identifier'' atom which uniquely identifies the field being defined, within
the scope of its class and all of its superclasses.


\subsubsection{Field declarations}
\label{sec:syntax:field-declarations}

Each field declaration can come in one of three forms:
\begin{enumerate}
\item The first form is for regular fields, which are typed at the time they
  are declared.  Valid field types are discussed in \TBD
\item The second form is for implicitly typed fields, i.e.~fields for which
  the type will be inferred from its uses elsewhere in the program.
\item The last form is for fields which are implicitly typed by constraining
  them to take values only in the set of values given by some field in another
  class.  In this case the type is inferred to be the same as that of the
  target field.
\end{enumerate}

The \textsf{FIELD\_ATTRS} symbol stands for a possibly empty sequence of
Prolog terms which describe attributes specific to the field which immediately
precedes the attribute.  These attributes are taken from the set shown in
figure~\ref{fig:syntax:field-attr}.
\begin{figure}[htb]
  \centering
  \begin{eqnarray*}
    \mathsf{FIELD\_ATTRS} & \rightarrow & \mathsf{FIELD\_ATTRS} \;
    \mathsf{FIELD\_ATTR} \;|\; \varepsilon \\
    \\
    \mathsf{FIELD\_ATTR} & \rightarrow & \token{unique} \; \token{.} \\
     & | & \token{not\_null} \; \token{.} \\
     & | & \token{key} \; \token{.} \\
     & | & \token{key} \; \mathsf{list}(\mathsf{FIELD\_NAME}) \; \token{.} \\
     & | & \token{unique} \; \token{.} \\
     & | & \token{domain} \; \mathsf{CLASS\_NAME} \token{.}
    \mathsf{FIELD\_NAME} \token{.}
  \end{eqnarray*}
  \caption{ISCO field attribute definition syntax}
  \label{fig:syntax:field-attr}
\end{figure}

\subsubsection{Remainder of a class declaration}
\label{sec:rema-class-decl}

Classes may be suffixed with attributes taken from the set specified by the
\textsf{CLASS\_ATTR} production: this is simply an alternate syntax and adds
nothing to the semantics.

\textsf{computed} classes are expected to contain one or more \emph{rules},
which come after the body of the class.  These rules are regular Prolog
clauses, with the following differences:
\begin{itemize}
\item The clause head is always ``\texttt{rule :-}''.
\item The clause body may access implicit head variables, which are named
  after the fields in the class definition, rewritten in all-upper-case.
\item ISCO predicate calls are subject to the preprocessing discussed in~\TBD
\end{itemize}


\subsubsection{Sequence declaration}
\label{sec:sequence-declaration}

One of the SQL features that is most useful in building applications is the
\texttt{sequence} construction.  ISCO incorporates this concept with minimal
impact on the language structure: sequences are declared in a syntactic form
similar to that of regular classes and accessed through a pair of predicates
which perform the sequence operations (fetch or set next value and inquire
aboutr the current value.)

\begin{figure}[htb]
  \centering
  \begin{eqnarray*}
    \mathsf{SEQUENCE\_DEF} & \rightarrow &  \\
  \end{eqnarray*}
  \caption{ISCO sequence definition syntax}
  \label{fig:syntax:sequence}
\end{figure}

\begin{figure}[htb]
  \centering
  \begin{eqnarray*}
    \mathsf{EXTERNAL\_DEF} & \rightarrow &
    \keyword{external}\token{(}\mathsf{ID}\token{,}
    \mathsf{EXT\_DESCRIPTOR}\token{)} \token{.}
  \end{eqnarray*}
  \caption{ISCO external data source/sink definition syntax}
  \label{fig:syntax:external}
\end{figure}

\begin{figure}[htb]
  \centering
  \begin{eqnarray*}
    \mathsf{list}(X) & \rightarrow & \token{[} \; \mathsf{sequence}(X) \;
    \token{]} \\
    \mathsf{sequence}(X) & \rightarrow & \mathsf{sequence}(X, \token{,}) \;|\;
    \varepsilon \\
    \mathsf{sequence}(X, \emph{SEP}) & \rightarrow & X \;|\;
    X \; \emph{SEP} \; \mathsf{sequence}(X, \emph{SEP})
  \end{eqnarray*}
  \caption{Utility syntax}
  \label{fig:syntax:utility}
\end{figure}

\subsection{ISCO goal syntax}
\label{sec:isco-goal-syntax}


% Local variables:
% mode: reftex
% mode: font-lock
% mode: auto-fill
% TeX-master: "manual"
% End:
